

\section{Data}

\begin{transitionframe}

    \rmfamily % Font
    
    \begin{center}
    {\Huge \textbf{\textcolor{white}{Data}}}
    \end{center}
  
\end{transitionframe}

\begin{frame}

    \frametitle{Data} % Title
    \framesubtitle{}  % Subtitle
    \rmfamily % Font

    \begin{wideitemize}
        \item \textcolor{fgre}{USBLS's LAUS Database} \(\to\) county-level unemployment statistics (1990-2023)
        \item \textcolor{fgre}{US Census Bureau's Population Estimates Program (PEP)} \(\to\) county-level population and demographics (1969-2020)
        \item \textcolor{fgre}{USDOL's Changes in Basic Minimum Wages} (1968-2023)
        \item \textcolor{fgre}{Horwitz et al. (2020)} \(\to\) Prescription Drug Monitoring Programs (1990-2019)
    \end{wideitemize}

    \vspace{9pt} 
    Current working sample:
    \vspace{9pt}
    
    \begin{wideitemize}
        \item Unemployment, PDMPs, Minimum Wage rates and demographics for counties of 50 States + DC, 1998-2019
    \end{wideitemize}

\end{frame}


\begin{frame}

    \frametitle{PDMPs} % Title
    \framesubtitle{}  % Subtitle
    \rmfamily % Font
    
    Prescription Drug Monitoring Programs (\textcolor{fblu}{PDMPs}): electronic database that tracks controlled substance prescriptions
    \vspace{9pt}
    
    \begin{wideitemize}
        \item \textcolor{fblu}{Modern System PDMPs}: the electronic database becomes accessible to any authorized user (eg, physician, pharmacist, or member of law enforcement)
        \item \textcolor{fblu}{Must Query PDMPs}: the law mandates the prescriber to check the database before prescribing a listed opioid
    \end{wideitemize}
    
\end{frame}