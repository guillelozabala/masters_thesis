\section{Model}

\begin{transitionframe}

    \rmfamily % Font
    
    \begin{center}
    {\Huge \textbf{\textcolor{white}{Model}}}
    \end{center}
  
\end{transitionframe}

\begin{frame}

    \frametitle{Set Up} % Title
    \framesubtitle{}  % Subtitle
    \rmfamily % Font

    An addicted individual faces
    \vspace{9pt}
    \begin{wideitemize}
        \item A health/performance \textcolor{fblu}{cost of opioid consumption} \(O_{it}\)
        \item This cost depends on \textcolor{fblu}{legal opioids consumption} \(l_{it}\) and \textcolor{fblu}{black market opioids consumption} \(h_{it}\) (say, heroin)
        \item Which opioid is consumed depends on the \textcolor{fblu}{availability of legal opioids} \(z_{t}\) (\(= 1\) if prescription law is in place)
        \item I assume that opioid misuse impedes the worker's ability to work
        %\item I assume that illegal opioids are more harmful than legal opioids 
        %\[h_{it} = l_{it} + \mathcal{C}\]
    \end{wideitemize}

\end{frame}

\begin{frame}

    \label{decision_making}
    \frametitle{Decision making} % Title
    \framesubtitle{}  % Subtitle
    \rmfamily % Font

    \begin{wideitemize}
        %\item The \textcolor{fblu}{total cost of opioid consumption} is
        \item The \textcolor{fblu}{decision rule of opioid consumption} is
        \[
        O_{it} = z_{t}\left(1-q_{it}\right)h_{it} + (1-z_{t})l_{it}
        \]
        where \(q_{it} = 1\) \textcolor{fblu}{if she quits opioids}
        \item The worker quits if
        
        \[
        q_{it} = \mathbbm{1}\{\text{Exp. utility of quitting} \geq \text{Disutility of quitting}\}
        \]
        \vspace{-15pt}
        \item Quitting and looking for a job are equivalent
    \end{wideitemize}
    %\hyperlink{quitting}{\beamerbutton{Quitting}}    

\end{frame}

\begin{comment}
\begin{frame}
2
    \frametitle{Labor demand} % Title
    \framesubtitle{}  % Subtitle
    \rmfamily % Font

    \begin{wideitemize}
        \item The \textcolor{fblu}{representative firm} maximizes profits
        \[
        \pi_{it} = Y_{it} - w_{it}L_{it} \quad \text{s.t} \quad w_t \geq w^{min}_t
        \]
        \vspace{-15pt}
        \item The \textcolor{fblu}{MPL} is given by
        \[
        \dfrac{\partial Y_{it}}{\partial L_{it}} = f(\theta_{it}) - \kappa\cdot\mathbbm{1}\{O_{it}\geq o\}
        \]
        where \(\theta_{it}\) is the \textcolor{fblu}{worker's ability} and \(\kappa\) captures the \textcolor{fblu}{impact of addiction on productivity}
    \end{wideitemize}

\end{frame}
\end{comment}

\begin{frame}

    \frametitle{Labor market} % Title
    \framesubtitle{}  % Subtitle
    \rmfamily % Font

    \begin{wideitemize}
        \item The market has a \textcolor{fblu}{job finding rate} 
        \[
            p_{it} = m_{it}/u_{it} 
        \]
        where \(m_{it}\) is the \textcolor{fblu}{number of matches} and \(u_{it}\) is the \textcolor{fblu}{number of unemployed workers}
        
        \item Matches are an increasing function of vacancies \(v_{it}\)
        \[
            m_{it} = M(v_{it}, u_{it})
        \]
        where \(M\left(\cdot \right) \) is a matching function
        \item I assume that more vacancies are open where the minimum wage is less binding ({\footnotesize  \textcolor{fgre}{Flinn, 2010}})
    \end{wideitemize}

\end{frame}


\begin{frame}

    \frametitle{Quitting decision} % Title
    \framesubtitle{}  % Subtitle
    \rmfamily % Font

    \begin{wideitemize}
        \item It follows that
        \[
            p_{w_{it} > w^{min}_t} \geq p_{w_{it} = w^{min}_t}
        \]
        \vspace{-15pt}
        \item If \(\textcolor{fblu}{w^{min}_t}\) is not binding
        \[
            q_{w_{it} > w^{min}_t} = \mathbbm{1}\{p_{w_{it} > w^{min}_t}\cdot u(w_{it}L_{it}) \geq \mathcal{C}^{quit}\}
        \]
        \vspace{-15pt}
        \item Otherwise
        \[
            q_{w_{it} = w^{min}_t} = \mathbbm{1}\{p_{w_{it} = w^{min}_t}\cdot u(w^{min}_t L_{it}) \geq \mathcal{C}^{quit}\} 
        \]

    \end{wideitemize}
    
\end{frame}


\begin{frame}

    \frametitle{Impact on labor market outcomes} % Title
    \framesubtitle{}  % Subtitle
    \rmfamily % Font

    \begin{wideitemize}
        \item As long as
        \[
            p_{w_{it} > w^{min}_t}\cdot u(w_{it}L_{it}) > \mathcal{C}^{quit} > p_{w_{it} = w^{min}_t}\cdot u(w^{min}_t L_{it})
        \]
        we should see (after \(z_{t} = 1\))
        \vspace{9pt}
        \begin{wideitemize}
            \item[\textcolor{fblu}{\textbullet}] An increase in participation rates where the minimum wage is less binding (\(O_{it} = 0\))
            \item[\textcolor{fblu}{\textbullet}] An increase in illegal opioid consumption where the minimum wage is more binding (\(O_{it} = h_{it}\))
        \end{wideitemize}

    \end{wideitemize}
    
\end{frame}

\begin{comment}
\begin{frame}

    \frametitle{Impact on labor market outcomes} % Title
    \framesubtitle{}  % Subtitle
    \rmfamily % Font

    \begin{wideitemize}
        \item Assume for the moment that \textcolor{fblu}{the worker cannot quit opioids} and that \(l_{it} < o < h_{it}\)
        \item Then if the \textcolor{fblu}{minimum wage is binding} (\(w_{t} = w^{min}_t\)) and \textcolor{fblu}{the law is passed} (\(z_t = 1\))
        \[
        f(\theta_{it}) - \kappa < w^{min}_t \,\Rightarrow\, \text{unemployment}
        \]
        \vspace{-15pt}
        \item But if the \textcolor{fblu}{minimum wage is not binding} (\(w_{t} >> w^{min}_t\)) 
        \[
        f(\theta_{it}) - \kappa = w_t 
        \]
        \vspace{-15pt}
        \item What if the worker can quit opioids?
    \end{wideitemize}
    \hyperlink{quitting}{\beamerbutton{Quitting}} 

\end{frame}
\end{comment}
