\documentclass[12pt,a4paper]{article}
\usepackage[margin=1in]{geometry}
\usepackage{graphicx}
\usepackage{amsmath,amssymb,amsfonts,bbm} % For math
\usepackage{comment} % Comment blocks
\usepackage{hyperref}
\usepackage{setspace}
\usepackage{booktabs} % For tables
\usepackage{longtable}
\usepackage[backend=bibtex,style=authoryear]{biblatex}

\renewcommand{\baselinestretch}{1.25}

\addbibresource{refs.bib}

\hypersetup{
    colorlinks=true,
    linkcolor=blue,
    filecolor=magenta,      
    urlcolor=cyan,
    citecolor=cyan,
    pdftitle={paper},
    pdfpagemode=FullScreen,
}

\setlength{\abovedisplayskip}{3pt}
\setlength{\belowdisplayskip}{3pt}

\setlength\parindent{0pt}
\addtolength{\parskip}{2mm}

\title{The Opioid Crisis: State Regulations and Labor Market Outcomes \\
[2ex] \large CEMFI}
\author{Guillermo Martínez \\ [2ex] Advisor: Tom Zohar}
\date{\today}

\begin{document}

\maketitle

\begin{abstract}
    \noindent
    While the effects of opioid misuse on the labor market have been studied in detail, little is known about how labor market regulations, particularly the minimum wage, shape this relationship. Do minimum wage rates affect the relationship between opioid misuse and labor market outcomes? To tackle this question, I use the staggered rollout of Must Query Prescription Drug Monitoring Programs (PDMPs) as a source of variation. These policies restrict prescription opioids for those considered to be misusing them, forcing individuals to either quit consumption or switch to illegal alternatives. The benefits of quitting opioids depend on the labor market opportunities available to those who withdraw. Through their negative impact on labor demand, high (binding) minimum wages can reduce the incentives to avoid opioid consumption. I find that labor force participation and unemployment rates increased only where the minimum wage was not binding, while a surge in heroin consumption followed where it was. These results highlight the importance of considering labor market regulations when studying the effects of the opioid crisis on the labor market.
\end{abstract}

\newpage

\section*{Introduction}

For the past two decades, the United States has been facing the worst public health crisis in its recent history.
The so-called opioid epidemic, characterized by a rising number of deaths due to opioid overdose, has caused more than 680,000 fatalities between 1999 and 2022 (\cite{CDCMCD9920}, \cite{CDCMCD1822}), mostly among non-college-educated Americans.
The sheer magnitude of this crisis has significantly impacted the labor market, prompting economists to study the relationship between opioid misuse and labor market outcomes.
However, little is known about how labor market regulations have shaped this relationship.
Minimum wage rates emerge as a natural candidate for studying this dynamic since they affect the same subpopulation of low-educated workers and have increased with varying intensity during the same period.

%This situation has prompted economists to study the crisis's effects on the labor market.
%Do minimum wage rates affect the relationship between opioid misuse and labor market outcomes?

\begin{comment}
    MORITZ:
    What do you mean by recent? % common in the literature
    Why so-called? 
    Is it important that you mention non-college educated people NOW?
    I think you need a better connection from opioid deaths to the study of labor market regulations
    I am not sure you have to talk about labor market regulation in general terms at all, you can immediately go into minimum wage rates that is very interesting in its own right % I still need regulations to give a better intuition
    I think I know the connection between low education levels and minimum wage rates (based on bindingness) but the connection here could be clearer
    I have questions about the importance of access to opioids in driving the epidemic
\end{comment}

\begin{comment}
    This paper argues that minimum wage rates affect the relationship between opioid misuse and labor market outcomes.
    To measure this relationship, I use the staggered rollout of Must Query Prescription Drug Monitoring Programs (PDMPs) across states.
    These programs require authorized users to check a patient's prescription history before prescribing opioids.
    By effectively combating opioid misuse, these measures force those addicted to either quit opioid consumption or switch to illegal alternatives.
    The incentives to choose one option over the other depend on the labor market opportunities available to those who withdraw. 
    Quitting opioids, although costly and often painful, offers benefits not only in better health outcomes but also in job prospects. 
    However, for those facing a minimum wage high enough to lower labor demand (binding), the benefits of quitting opioids might be lower than for those with greater job opportunities.
    My estimations support this intuition, showing increases in labor force participation and unemployment rates as a result of prescription policies only where the minimum wage was not binding, while a surge in heroin consumption occurred where it was.
\end{comment}

This paper argues that minimum wage rates influence the relationship between opioid misuse and labor market outcomes.
I measure this relationship using the staggered rollout of Must Query Prescription Drug Monitoring Programs (PDMPs) across states, which require authorized users to check a patient's prescription history before prescribing opioids. 
These programs force addicts to either quit opioids or switch to illegal alternatives, with the choice depending on available labor market opportunities.
Quitting opioids, though difficult, improves health and job prospects.
However, in areas with high (binding) minimum wages, reduced labor demand might lower the benefits of quitting.
My estimations show increases in labor force participation and unemployment rates only where the minimum wage was not binding, while a surge in heroin consumption occurred where it was.

%at the time the law is passed

% Condensate the whole background part into a single (two) paragraph maybe not mention lit yet

These findings contribute to the growing body of literature studying the economic effects of the opioid crisis.
Within this literature, the origins of the epidemic have been widely discussed, with a supply-side view of its origins now largely accepted over a demand-side explanation.
This perspective has led economists to focus on supply-side interventions to explore the causal relationship between opioid consumption and labor market outcomes.
While well-intentioned, many of these policies have had unintended consequences, such as the rise in heroin consumption following the reformulation of OxyContin.
However, Prescription Drug Monitoring Programs (PDMPs), have been less scrutinized.
Recent research indicates that, in their more stringent form, these programs can effectively reduce prescription opioid misuse.

\begin{comment}
    These findings contribute to the growing body of literature studying the economic effects of the opioid crisis and of the policies implemented to curb it.
    Within this literature, the origins of the epidemic have been widely discussed, with a supply-side view of its origins now largely accepted over a demand-side explanation.
    This perspective has led economists to focus on supply-side interventions to explore the causal relationship between opioid consumption and labor market outcomes.
    Prescription Drug Monitoring Programs (PDMPs) 
    Recent research indicates that, in their more stringent form, these programs can effectively reduce prescription opioid misuse.
\end{comment}

Yet, how heterogenous these effects are across labor markets remains an open question.
Different incentives can lead victims of opioid misuse to take different decisions regarding both their health and their economic prospects. 
Labor market regulations, by altering the labor outcomes of workers, influence this decision-making process by affecting the incentives to quit opioids -- a family of drugs that can severely impair the ability to work.
To understand how these regulations have shaped the ongoing crisis, I focus on minimum wage rates, which have increased in all states at different speeds during the same period as the opioid crisis, thereby affecting the labor market opportunities of low-educated workers.
Critical to our argument are not the rates themselves, but how binding they are across different labor markets: the more binding the minimum wage, the less incentives workers have to quit opioids, as they face a tougher job market search.

I use several data sources to obtain a distribution of coefficients of the impact of opioid misuse on labor market outcomes across different minimum wage bindingness levels.
The US Bureau of Labor Statistics (BLS) LAUS database provides county-level labor market statistics, while the US Census Bureau's Population Estimates Program (PEP) offers county-level population and demographic data.
From \textcite{Horwitz2021} I obtain data on Prescription Drug Monitoring Programs (PDMPs). 
The BLS's Occupational Employment and Wage Statistics (OEWS) database provides national industry-specific occupations and wages statistics, \textcite{Eckert2020} offers county-level industry-specific employment data, and the US Department of Labor's Changes in Basic Minimum Wages provides information on minimum wage rates; all of these data are relevant for constructing a measure of minimum wage bindingness.

The Centers for Disease Control (CDC) National Vital Statistics System offers data on total drug poisoning mortality by county, while the CDC's Multiple Cause of Death database provides overdose deaths at the state-year level for different drugs.
Finally, the Drug Enforcement Administration's Automation of Reports and Consolidated Orders System (ARCOS) offers opioid distribution data.
The joint panel covers the United states for the period 2003-2016.

The results are striking in several dimensions.
After constructing a measure of minimum wage bindingness known as the Kaitz-\(\rho\) index, I find that the impact of prescription policies on labor market outcomes is indeed heterogeneous across different levels of minimum wage bindingness.
First, I find that policies aimed at stopping opioid misuse increased participation rates only where the minimum wage was not binding.
Secondly, the increase was short-lived (less than two years), but the impact on unemployment rates was persistent.
Moreover, their impact on prescription rates seem to be the main, but not the only, driver of these results.
Finally, prescription policies had no labor market consequences where where the minimum wage was binding, but their impact on heroin deaths was increasing in minimum wage bindingness.

Therefore, this paper sheds light on how labor market regulations, and in particular minimum wage rates, give rise to heterogeneities in the effects of the opioid crisis on the labor market, contributing to a better understanding of the ongoing crisis and the policies implemented to address it.
More broadly, it contributes to the understanding of how labor market regulations interact with drug disorders.

The paper is structured as follows.
First, it provides a summary of the opioid crisis and the effects it has had on the labor market as uncovered by the literature.
Then, it discusses some of the policies implemented to curb the opioid misuse and its effectiveness.
Later, I describe how different minimum wage rates can explain some of these policy consequences and the heterogeneity of the labor market outcomes. 
A model is introduced to formalize the intuition behind this argument.
The employed data is discussed before presenting the empirical strategy and obtained results.


\section*{Background: Opioid Crisis and regulatory changes}

% CHECKED
In this section, I summarize the origins of the opioid crisis and its general effects on the labor market.
I then discuss various policies implemented to curb opioid misuse, and how some of them resulted ineffective or counterproductive. 
Finally, I explain how labor market incentives, in particular minimum wage rates, can account for some of these policy failures and the heterogeneity in labor market outcomes.

\subsection*{The origins of the opioid epidemic and its impact on the labor market}

% CHECKED
It is difficult to overstate the importance of the opioid crisis in the United States. 
In 2022 alone, 110,000 people died from drug abuse, three-quarters of them due to opioids.
For context, the Vietnam War -- one of the most traumatic events in recent American history -- killed 58.000 US soldiers in total.
Given the severity of the crisis, it is not surprising that a large body of literature has emerged to study its origins and consequences.

% CHECKED
Although demand-side explanations of the crisis exist, with \textcite{Deaton2017} and \textcite{Deaton2015} being some of the most influential, a supply-side view of the epidemic's origins is now widely accepted (\cite{Eichmeyer2022}, \cite{Ruhm2018}). 
This shift is partly due to mixed evidence on the causal impact of declining economic conditions on opioid consumption (\cite{Currie2018}).
Moreover, the efforts by pharmaceutical companies and healthcare providers to increase opioid prescriptions since the late 1990s have been well documented (\cite{Alpert2019}, \cite{Kolodny2015}, \cite{VanZee2008}).
Purdue Pharma's aggressive marketing campaign of OxyContin, a powerful opioid painkiller, is often cited as the starting point of the crisis (\cite{Chow2019}).

% CHECKED
Economists quickly began studying the effects of the opioid crisis on the labor market.
In a seminal paper, \textcite{Krueger2017} found that labor force participation rates were falling on greater numbers in areas with higher opioid prescription rates, an analysis later refined by \textcite{Aliprantis2023}.
This relationship has also been emphasized from a macroeconomic perspective (\cite{Greenwood2022a}, \cite{Greenwood2022b}).
Many researchers have used exogenous variations in prescription rates to further establish the causal relationship between opioid consumption and worsened labor market outcomes (\cite{Beheshti2023}, \cite{Powell2022}, \cite{Harris2020}).
These papers typically rely on federal- or state-level regulatory changes implemented in response to the ongoing public health crisis.
However, none have delved into how labor market regulations can create heterogeneity in these effects.

\subsection*{Policy efforts and unintended consequences}

% CHECKED
Policy efforts to curb the opioid crisis have been numerous and varied, although not all of them have been equally effective.
A much-discussed example is the reformulation of OxyContin in 2010, which made the dosage units more difficult to crush or dissolve, thus reducing the drug's abuse potential through non-oral routes.
This change led to a surge in heroin consumption as users switched to a more readily accessible alternative for misuse (\cite{Alpert2018}, \cite{Evans2019}, \cite{Cicero2012}, \cite{Butler2013}).
Additionally, the switch to heroin seemed to result in a greater prevalence of blood-borne diseases in the most affected areas (\cite{Beheshti2019}).
Lastly, there is anectodic evidence that those who did not choose a black market alternative still found ways to make the drug injectable (\cite{Cicero2015}).

% CHECKED
Abuse-deterrent formulations were not the only efforts that backfired. 
For instance, \textcite{Doleac2022} find that increased access to Naloxone (a drug used to reverse opioid overdoses) led to a rise in opioid-related emergency room visits and opioid-related crime. 
The impact of these policies on opioid-related mortality seems to be restricted to early-adopter states (\cite{Rees2019}).
Medical marijuana laws (\cite{Powell2018}), Medicare Part D (\cite{Powell2020}), substance abuse treatment availability (\cite{Swensen2015}), and physicians' education (\cite{Schnell2018}) are among the many policies and factors that economists have studied in relation to the opioid epidemic, contributing to a vast body of literature. 

% CHECKED
Fewer concerns have been raised about the unintented consequences of Prescription Drug Monitoring Programs (PDMPs).
According to the \textcite{CDC2021}, a PDMP is ``\textit{a database that collect information about dispensed prescription drugs.}''.
The collected data can only be accessed by authorized users (e.g., physicians, pharmacists, or law enforcement members) and is used to identify and prevent drug misuse.
Although early studies questioned the effectiveness of these programs in reducing opioid misuse (\cite{Paulozzi2011}, \cite{Maughan2015}, \cite{Meara2016}) and the effectiveness of supply-side interventions in drug-related policies is mixed (\cite{Pollack2014}, \cite{Dobkin2019}, \cite{Dobkin2014}), more recent work has found that in states where prescribers are mandated to check the database before prescribing a listed opioid, the number of opioid misuse cases related to prescription opioids decreased (\cite{Buchmueller2018}, \cite{Sacks2021}, \cite{Grecu2019}).
These more stringent PDMPs, known as Must Query PDMPs, therefore offer a unique opportunity to study the relationship between opioid misuse and labor market outcomes.
Their staggered rollout across states, combined with the variation at the county level in how binding minimum wage rates are, allows for a quasi-experimental design that can shed light on how labor market regulations shape the effects of the opioid crisis.

\subsection*{Bringing minimum wage rates in}

% CHECKED
During the same period as the Opioid Crisis, minimum wage rates have been increasing in all states, albeit at different speeds.
While some states such as California and New York have raised their minimum wage to over \$15 per hour, others have kept it at the federal minimum of \$7.25.
This spatial and temporal variability makes minimum wage rates a natural candidate to study the mediation of labor market regulations on the relationship between opioid misuse and labor market outcomes. 
However, its relevance hinges on whether these minimum wage rates affect labor demand.

% CHECKED
Although the impact of minimum wages on labor demand has been questioned since \textcite{Card1994} published their seminal paper, recent research indicates that minimum wage increases can negatively affect employment in certain contexts.
For instance, we know that minimum wage increases can lead to a decrease in employment in tradeable sectors (\cite{Cengiz2019}), and in industries where passing costs to consumers is more difficult (\cite{Harasztosi2019}).
This impact seems to be more pronounced for low-skilled workers (\cite{Neumark2004}), although the effect is negligible for small increases (\cite{Clemens2021}).
The macroeconomic situation can also influence the effect of minimum wage increases on employment if the former is particularly binding (\cite{Clemens2019}). 
However, \textcite{Engbom2022} noted that these effects can be muted by reallocations of workers across firms.
Crucially, they argue that lower-productivity firms reduce vacancy creation and hiring, which in our context could lead to lower job finding rates where the minimum wage is more binding.

% CHECKED
If relatively high minimum wages reduce labor demand for low-skilled workers, or reduce it in areas where lower-productivity firms are more prevalent, the prospective labor earnings of non-college-educated workers should be lower in these areas.
This could, in turn, reduce the incentives for opioid addicts to withdraw as prescription drug regulations are passed. 
The difficult and often painful process of withdrawal might be less appealing compared to opting for illegal alternatives if job opportunities are scarce.
This intuition, which I formalize in the next section, suggests that the effects of prescription drug regulations on labor market outcomes should be more pronounced in areas where the minimum wage is not binding.

\section*{A model of job searching and opioid consumption}

The following model formalizes the intuition outlined above. 
An addicted worker \(i\) faces a health/performance cost of opioid consumption \(O_{it}\) at time \(t\).
This cost depends on whether she consumes prescription/legal opioids \(l_{it}\) or illegal opioids \(h_{it}\) (e.g., heroin).
The type of opioid consumed depends on the availability of legal opioids \(z_{it}\), which equals 1 if an effective prescription law is in place.
I assume that opioid misuse impedes the individual's ability to work.

The decision rule of opioid consumption is hence given by:

\begin{equation}
    O_{it} = z_{t}\left(1-q_{it}\right)h_{it} + (1-z_{t})l_{it}
\end{equation}

where \(q_{it} = 1\) if she quits opioids. 
Quitting and looking for a job are equivalent in this setting.

From (1), we see that the determinants of the decision to quit are crucial to understanding the effects of a prescription law.
This decision depends on the labor market opportunities available to the individual.
The addicted individual faces a labor market with a job finding rate

\begin{equation}
    p_{c(i)t} = \dfrac{m_{c(i)t}}{u_{c(i)t}}
\end{equation}

where \(m_{c(i)t}\) is the number of matches and \(u_{c(i)t}\) is the number of unemployed workers in the county the agent lives, \(c(i)\). 
I assume that matches are a non-decreasing function of vacancies \(v_{c(i)t}\)

\begin{equation}
    m_{c(i)t} = M(v_{c(i)t}, u_{c(i)t})
\end{equation}

where \(M\left(\cdot \right) \) is a matching function. 
Following \textcite{Flinn2010}, I assume that more vacancies are open where the minimum wage is less binding. 
Then, it is easy to see that

\begin{equation}
    p_{\{w_{it} > w^{min}_t\}} \geq p_{\{w_{it} = w^{min}_t\}}
\end{equation}

that is, the job finding rate the individual faces is higher if the minimum wage is not binding.

The decision to quit opioids is hence influenced by the minimum wage as follows:
If the individual quits, she will seek a job that provides utility \(u(w_{it}L_{it})\), where \(w_{it}\) is the wage she receives and \(L_{it}\) is the number of hours she works.
The probability of finding a job is given by the job finding rate.
If the product of the utility and the job finding rate is greater than the disutility of quitting, \(\mathcal{C}^{quit}\), the individual will quit.
Formally

\begin{equation}
    q_{it}  = \mathbbm{1}\{p_{{c(i)t}} \cdot u(w_{it}L_{it}) \geq \mathcal{C}^{quit}\}
\end{equation}

Unless a corner solution exists (i.e., the disutility of quitting is greater or lower than the maximum utility the individual can get from working), there must be some wage level such that

\begin{equation}
    p_{\{w_{it} > w^{min}_t\}}\cdot u(w_{it}L_{it}) > \mathcal{C}^{quit} > p_{\{w_{it} = w^{min}_t\}}\cdot u(w^{min}_t L_{it})
\end{equation}

that is, for some wage level, the job finding rate ensures that the utility from working exceeds the disutility of quitting.
Notice that this is irrespective of the differences in levels of wages, since the job finding rate is also playing a role.

If condition (6) holds, after a prescription law is enacted (\(z_{t} = 1\)) we should observe an increase in participation rates where the minimum wage is less binding (\(O_{it} = 0\)) but also an increase in illegal opioid consumption where the minimum wage is more binding (\(O_{it} = h_{it}\)).
We test these hypotheses in the next sections.

\section*{Data}

% CHECKED
To uncover the heterogeneity in the effects of the opioid crisis on the labor market created by minimum wage rates, I use several data sources.
The result is a panel of county-level data for the United States for the period 2003-2016 encompassing labor market, demographic and health variables.
Monthly data is utilized wherever available; otherwise, I use the data available for the year.
Similarly, state-level minimum wage rates are allocated to the counties.

% CHECKED
\textit{Labor market data and demographics}: 
I use the US Bureau of Labor Statistics (BLS) Local Area Unemployment Statistics (LAUS) database to access county-level labor market statistics. 
Specifically, I gather monthly data on labor force participation, employment, and unemployment rates at the month-county level.
This dataset covers the period 1990-2023. 
Additionally, through the US Census Bureau's Population Estimates Program (PEP), I obtain demographic information at the county-year level spanning from 1969 to 2020. 
This dataset provides the population counts of each county by gender, race and origin (Hispainc or Non-Hispanic), on a yearly basis.
Moreover, I can obtain the number of people in each of the 19 age groups considered, which is used to construct the labor force participation rate.

% CHECKED
\textit{Minimum wage bindingness}: 
I utilize the US Department of Labor's Changes in Basic Minimum Wages dataset to gather information on minimum wage rates at the state level from 1968 to 2023.
Estimates of the wage distribution at the county level are requiered to construct a measure of minimum wage bindingness, as outlined in the next section.
For this purpose, I rely on the US Bureau of Labor Statistics (BLS) Occupational Employment and Wage Statistics (OEWS) database to obtain national industry-specific occupations and wages statistics.
This dataset reports the number of workers in each occupation and the 10th, 25th, median, 75th and 90th wage percentiles for each industry, from 1997 to 2023.
Additionally, I use the data from \textcite{Eckert2020} to obtain county-level industry-specific employment data spanning from 1975 to 2016.
These last two datasets have an annual frequency. 4-digit NAICS codes are used to match the industries across datasets.

% CHECKED
\textit{Prescription Drug Monitoring Programs (PDMPs)}: 
I use data from \textcite{Horwitz2021} to obtain information on Prescription Drug Monitoring Programs (PDMPs) from 1990 to 2019.
This dataset offers state-year level detailing the implementation of PDMP types across states, as illustrated in Table 1. 
PDMPs are categorized into Modern System and Must Query types. 
A Modern System PDMP grants electronic database access to authorized users, whereas a Must Query PDMP mandates prescribers to consult the database before prescribing listed opioids. 
This last type of PDMP is the one most likely to have an impact on opioid misuse, and it forms the basis of my estimations.

% CHECKED
\textit{Prescriptions and overdose deaths}: 
I use data from the Centers for Disease Control (CDC) National Center for Health Statistics (NCHS) to obtain information on total drug poisoning mortality by county, from 2003 to 2021.
To overcome the suppression of data following anonimity concerns, \textcite{CLMort} employs Hierarchical Bayesian models with spatial and temporal random effects to estimate the number of deaths where the data is not available.
I further use the Multiple Cause of Death database (\cite{CDCMCD9920}), which reports overdose deaths at the state-year level for different drugs, from 1999 to 2020, to obtain information on overdose deaths related to heroin (T.40.1), methadone (T.40.2), other opioids (T.40.3) and other synthetic narcotics (T.40.4).
The codes reffer to the Multiple Causes of Death ICD-10 classification. 
Overdose deaths are identified using the Underlying Cause of Death ICD-10 codes X40-X44, X60-X64, X85, and Y10-Y14.

% CHECKED
Finally, I use the Drug Enforcement Administration's Automation of Reports and Consolidated Orders System (ARCOS) to obtain opioid distribution data spanning from 2006 to 2014.
This dataset records the distribution of all Schedule II substances by active ingredient at the 3-digit ZIP code level.
The data is aggregated on semi-synthetic opioid analgesics (i.e. oxycodone, hydrocodone) prescribed at the county-month level.
The shorter time span does not affect the main specifications, since I use prescription rates only to assure that the PDMPs are effective in a complementary exercise.

\section*{Findings: how minimum wages shaped the economic effects of the opioid epidemic}

This section presents the main results of the paper. 
I start describing the construction of the Kaitz-\(\rho\) index, which is used to measure the bindingness of the minimum wage.
Then, I present the main especification of the paper, which is employed to estimate the effects of Prescription Drug Monitoring Programs on labor force participation and unemployment rates.
The coefficients of the regression are compared across the spatial distribution of minimum wage bindingness measures.
Finally, I repeat the same exercise using overdose deaths for different opioids as the dependent variable.

\subsection*{The Kaitz-\(\rho\) index}

The argument outlined above hinges on the assumption that the minimum wage can affect the labor market opportunities of low-educated workers.
In other words, at least in some areas it must be the case that the minimum wage is an effective, or binding, constraint on the labor market.
A common approach to measure this bindingness is through the Kaitz-\(\rho\) index, which can be defined as

\begin{equation}
    \text{\textit{kaitz}}_{ct}(\rho) \equiv \log{w^{\text{\textit{min}}}_{s(c)t}} - \log{w^{\rho}_{ct}}
\end{equation}

where \(w^{\text{\textit{min}}}_{s(c)t}\) is the minimum wage in the state where the county \(c\) is located at time \(t\), and \(w^{\rho}_{ct}\) is the \(\rho\) percentile of the wage distribution.
This measure generates an spatial distribution of the bindingness of the minimum wage, which can be used to compare the effects of the opioid crisis across different labor markets.
Figure 1 shows the distribution of the Kaitz-\(\rho\) index across counties in the United States for the year a Must Query PDMP is implemented in their states, for each of the different percentiles of the wage distribution included in the panel.
For the main results, I use the 10th percentile, but the findings are similar across the distribution -- the counties for which the minimum wage is more binding with respect to lower percentiles tend to be the same as those for which it is more binding with respect to higher percentiles.

The indices are obtained using the data on the wage distribution at the occupation-industry level, and the shares of employment of each industry in each county. 
In particular, let \(w^{\rho}_{oit}\) be the percentile \(\rho\) of the wages at the occupation \(o\) in the industry \(i\) at time \(t\). 
Then

\begin{equation}
    w^{\rho}_{it} = \sum_{o\in i} \dfrac{e_{oi}}{E_{it}}\,w^{\rho}_{oit}
\end{equation}
    
where \(e_{oi}\) is the number of workers in the occupation \(o\) and \(E_{it}\) is the total number of workers in the industry \(i\) at time \(t\).
That is, by weighting the wage percentiles by the share of employment of each occupation in the industry, I obtain the wage percentile for the industry.
After that, I compute

\begin{equation}
    w^{\rho}_{ct} = \sum_{i\in c} \dfrac{e_{ci}}{E_c}\,w^{\rho}_{it}
\end{equation}

where \(e_{ci}\) is the number of workers in the industry \(i\) in the county \(c\) and \(E_c\) is the total number of workers in the county.
\(w^{\rho}_{ct}\) is the wage percentile for the county.
Finally, the minimum wage rates at the state level are already included in the panel.

\subsection*{Estimation}

The main specification computes individual-level effects following \textcite{Arkhangelsky2024} and has the form

\begin{equation}
    y_{ct} = \alpha_c + \lambda_t + \sum_{k\geq h}\tau_{ck} + \beta X_{ct} + \nu_{ct}
\end{equation}

Here, \(y_{ct}\) is a labor market outcome, either the labor force participation rate or the unemployment rate, \(\alpha_c\) are county fixed effects, and \(\lambda_t\) are time fixed (year-month) effects.
Relative time is represented with \(k = t - z_c\), the time difference of the calendar time from the implementation of a PDMP in the state, represented by \(z_s\).
\(\tau_{ck}\) are the coefficients of interest and \(\nu_{ct}\) is an idiosyncratic shock.

The covariates \(X_{ct}\) include demographic shares (race, origin, sex and age), sector shares of employment (at the 2-digits NAICS codes), and total drug poisoning mortality, which might be a relevant variable in policy-making.
Crucial for identification, I assume that the state-level rollout of the PDMPs is exogenous at the county level after conditioning on the covariates.
Moreover, this strategy requires that there is no anticipation of the policy up to \(h\) periods. 
I assume no anticipation whatsoever, so that the coefficients are identified from the very start of the policy.
In this context, this means that no addicted individual decided to look for a job or switch to heroin because of the expected implementation of the PDMP. 

After running (10), a distribution of individual (county-level) coefficients is obtained.
These coefficients are then compared across the spatial distribution of the Kaitz-\(\rho\) index at the moment of treatment.
The comparison is made by running a binscatter following \textcite{Cattaneo2024}, which allows for an optimal binning of the data.

\subsection*{Labor force participation rates}

Using the labor force participation rate as the dependent variable, the results of the main specification are displayed in Figure 2.
The figure shows the distribution of the coefficients of the impact of the passing of the Must Query PDMPs on labor force participation rates across different levels of minimum wage bindingness, for different periods relative to treatment.
In the upper right panel we can see the immediate impact of the policy.
For those counties where the minimum wage is less binding, the effect is positive (1.37 p.p.) and significant (0.513, 2.22) at conventional levels.
The effects become smaller and more uncertain as the minimum wage becomes more binding.
For those counties in which the minimum wage has a higher impact on labor demand, the point estimate becomes negative (-0.693) but not significant (-2.54, 1.16).

The effect seems to be persistent in the short run (upper right and lower right panels), but fades away after two years (lower right)\footnote{The confidence interval of the most binnding counties is not displayed in this last panel for visualization purposes. The values are (-3.10, 1.90)}.
Figure 3 shows the time averages of the coefficients for two years after the law is passed.
The results remain positive and significant for those counties in which the minimum wage is less binding (1.06 p.p., (0.584, 1.55)), while they are not significant for those where the minimum wage was more binding.

The results seem to be driven mostly by the impact of the policy on prescription rates.
Although the overlap between the passing of the policies (Table 1) and the data for prescription rates (2006-2014) is scarce, we can still obtain som estimates on the impact of Must Query PDMPs on prescription rates per capita.
Figure 4 runs the model (10) with the prescription rates per capita as the dependent variable.
The two months after the treatment is passed are displayed.
We see that the policy is able to decrease the amount of prescriptions, with the biggest drops occuring where the minimum wage is the least and the most binding (the effects in these areas are not statistically different). 
Although the target of these policies where not to decrease prescriptions \textit{per se}, but only tackling misuse cases, the results suggest that the policies are effective in reducing the amount of opioids prescribed on the aggregate.

Moreover, this seems to be the main channel through which the policies affect labor market outcomes: Figure 8 in the appendix shows that the relation between labor market outcomes and minimum wage bindingness is not significant when controlling for prescription rates.
Figures 9 and 10 in the Appendix show the the point estimates of the binscatter for different Kaitz indices, for six months and two years after the policy is passed.
It is easy to see that the results do not change much across different measures, this being especially true the closer in time the treatment was implemented.

\begin{comment}
After six months:
1.79 (0.955, 2.63)
0.333 (-0.665, 1.33)

After two years:
-0.118 (-1.66, 1.42)
-0.599 (-3.10, 1.90)

Time averages:
1.06  (0.584, 1.55)
0.182 (-0.335, 0.698)
\end{comment}

\subsection*{Unemployment rates}

The analysis is repeated using the unemployment rate as the dependent variable.
The results are displayed in the same fashion in Figure 5.
The distribution of the coefficients reveal that almost all the positive effect it was found in the labor force participation rates translated into an increase in the unemployment rate.
For those counties where the minimum wage is less binding, the effect is positive (1.02 p.p.) and significant (1.46, 0.575) at conventional levels. 
The point estimate is only 0.35 p.p. below the one it was found for the labor force participation rate.
Although a positive effect is also found for those counties where the minimum wage is more binding (0.640 p.p.), the effect is not significant (-0.0252, 1.31).

These results shouldn't be surprising, at least in the short run: if people with severe opioid addictions start looking for a job after the implementation of a PDMP, they might not be able to find one immediately even in those counties where the minimum wage is less binding.
Strikingly though, the impact on unemployment persists after two years.
The lower right panel shows that, even after the increase in labor force participation rates fades away, the increase in unemployment rates remains high (2.77 p.p.) and significant (1.87, 3.68) where the minimum wage is less binding.
Whether the causes of this persistence are related to opioid consumption (through, say, relapsing) or to characterisctics of the workers or the labor market (say, if these workers face more frictional unemployment) remains an open question.

\begin{comment}
    After six months:
    1.01   1.48   0.543
    -0.0981 0.199 -0.396

    After two years:
    2.77   3.68   1.87
    -0.0720 0.265 -0.409 
\end{comment}

The time average of the coefficients, displayed in Figure 6, confirm our previous result.
The effect of the policy on unemployment rates is positive and significant for those counties where the minimum wage is less binding (1.25 p.p., (0.97, 1.52)), while it is not significant for those where the minimum wage is more binding (0.025 p.p., (-0.11, 0.16)).
Again, the results seem to be driven by the impact of the policy on prescription rates, especially in the medium run: Figure 11 in the Appendix shows that the impact stops being significantly different from zero after one year once we control for prescriptions per capita.
Figures 12 and 13 in the Appendix show the binscatter for different Kaitz indices, for six months and two years after the policy is passed.
The results do not change substantially across different measures.

\subsection*{Overdose deaths}

Finally, and in accordance with the model outlined above, I show some evidence that the rollout of Must Query PDMPs has had a positive impact on the consumption of illegal opioids in those counties where the minimum wage was more binding.
For this purpose, I run the same estimation strategy as before, but using the number of overdose deaths per capita for different opioids as the dependent variable. 
The data is at the state-year level, and the number of total poisoning deaths is not used as a control variable.
Figure 7 displays the scatter plot of the time average of the coefficients on the Kaitz-0.10 index of the states, for the two years after the policy is passed.
The opioids included are heroin, methadone, other opioids and other synthetic narcotics, following the ICD-10 classification.

The coefficients are regressed on the Kaitz index and a constant. 
Tables 2-5 show the values of the regression coefficients for each of the different opioids.
Although admittedly weaker than the evidence presented in previous sections due to data constaints, it is interesting that the only marginally significant relation is found for heroin.
This suggest that the positive relation between black market switching and low labor demand is present, but we cannot assert this relationship with the same degree of certainty. 

It seems implausible that part of the effect is absorbed by the consumption of other opioids.
Methadone is often used as a treatment for opioid adiction and can be legally used under a doctor's supervision.
Other opioids include drugs such as morphine and codeine, but also semi-synthetic opioid analgesics (e.g. drugs such as oxycodone, hydrocodone, hydromorphone, and oxymorphone), some of them targeted by the PDMPs.
Finally, although the category of other synthetic narcotics includes fentanyl and its analogues, its use didn't used to be as widespread as it is today during the period considered.

\section*{Conclusion}

This paper contributes to the understanding of how labor market regulations, particularly minimum wage rates, shape the relationship between opioid misuse and labor market outcomes. By leveraging the staggered rollout of Must Query Prescription Drug Monitoring Programs (PDMPs) across states, this study provides a nuanced view of how labor market opportunities influence the incentives for opioid users to quit or switch to illegal alternatives.

The findings reveal that labor force participation and unemployment rates increased as a result of prescription policies only in areas where the minimum wage was not binding. In contrast, a surge in heroin consumption occurred where the minimum wage was binding, highlighting the complex interplay between labor market conditions and opioid misuse. These results underscore the importance of considering labor market regulations when evaluating the economic impacts of the opioid crisis.

Furthermore, this study adds to the growing body of literature that examines the economic effects of the opioid epidemic. While previous research has primarily focused on the direct consequences of opioid misuse on labor market outcomes, this paper emphasizes the role of labor market regulations in mediating these effects. The heterogeneity in the impact of prescription policies across different levels of minimum wage bindingness suggests that policies aimed at curbing opioid misuse must account for the broader economic context to be effective.

In summary, the analysis demonstrates that higher minimum wages, by reducing labor demand for low-skilled workers, can diminish the incentives for opioid users to quit, thereby exacerbating the shift towards illegal alternatives like heroin. These insights highlight the need for a comprehensive approach to addressing the opioid crisis, one that integrates labor market policies with public health interventions to effectively mitigate the epidemic's adverse effects on both health and economic outcomes.

Future research should continue to explore the interplay between labor market regulations and health crises, considering the potential unintended consequences of well-intentioned policies. By doing so, policymakers can design more effective interventions that address the root causes of public health crises while also supporting economic stability and growth.

[MOVING FORWARD]

\newpage

\printbibliography

\newpage

\begin{longtable}[c]{lcccc}
    \caption{Prescriber Must Query and Modern Operational PDMPs Dates by State, from Horwitz et al. (2021)} \\
    \toprule
    \textbf{State} & \textbf{PMQ Year} & \textbf{PMQ Month} & \textbf{MOP Year} & \textbf{MOP Month} \\
    \midrule
    \endfirsthead
    
    \caption[]{(continued)} \\
    \toprule
    \textbf{State} & \textbf{PMQ Year} & \textbf{PMQ Month} & \textbf{MOP Year} & \textbf{MOP Month} \\
    \midrule
    \endhead
    
    \midrule \multicolumn{5}{r}{Continued on next page} \\
    \endfoot
    
    \bottomrule
    \endlastfoot

    Alabama & 2017 & 3 & 2006 & 4 \\
    Alaska & 2018 & 6 & 2012 & 1 \\
    Arizona & 2017 & 10 & 2008 & 12 \\
    Arkansas & 2017 & 8 & 2013 & 5 \\
    California & 2018 & 10 & 2009 & 9 \\
    Colorado & - & - & 2008 & 2 \\
    Connecticut & 2015 & 10 & 2008 & 7 \\
    Delaware & - & - & 2012 & 8 \\
    District of Columbia & - & - & 2016 & 10 \\
    Florida & 2018 & 7 & 2011 & 10 \\
    Georgia & 2018 & 7 & 2013 & 5 \\
    Hawaii & 2018 & 7 & 2012 & 2 \\
    Idaho & - & - & 2008 & 4 \\
    Illinois & 2018 & 1 & 2009 & 12 \\
    Indiana & - & - & 2007 & 7 \\
    Iowa & 2018 & 7 & 2009 & 3 \\
    Kansas & - & - & 2011 & 4 \\
    Kentucky & 2012 & 7 & 1999 & 7 \\
    Louisiana & 2014 & 8 & 2009 & 1 \\
    Maine & 2017 & 1 & 2005 & 1 \\
    Maryland & 2018 & 7 & 2013 & 12 \\
    Massachusetts & 2014 & 12 & 2011 & 1 \\
    Michigan & 2018 & 6 & 2003 & 1 \\
    Minnesota & - & - & 2010 & 4 \\
    Mississippi & - & - & 2008 & 7 \\
    Missouri & - & - & - & - \\
    Montana & - & - & 2012 & 10 \\
    Nebraska & - & - & 2017 & 1 \\
    Nevada & 2015 & 10 & 2011 & 2 \\
    New Hampshire & 2016 & 5 & 2014 & 10 \\
    New Jersey & 2015 & 11 & 2012 & 1 \\
    New Mexico & 2013 & 2 & 2005 & 8 \\
    New York & 2013 & 8 & 2013 & 6 \\
    North Carolina & 2017 & 6 & 2007 & 7 \\
    North Dakota & 2018 & 1 & 2008 & 10 \\
    Ohio & 2015 & 4 & 2006 & 10 \\
    Oklahoma & 2015 & 11 & 2006 & 7 \\
    Oregon & - & - & 2011 & 9 \\
    Pennsylvania & 2015 & 6 & 2016 & 8 \\
    Rhode Island & 2016 & 6 & 2012 & 9 \\
    South Carolina & 2017 & 5 & 2008 & 2 \\
    South Dakota & - & - & 2012 & 3 \\
    Tennessee & 2013 & 4 & 2010 & 1 \\
    Texas & 2019 & 9 & 2012 & 8 \\
    Utah & 2018 & 5 & 2006 & 1 \\
    Vermont & 2013 & 11 & 2009 & 1 \\
    Virginia & 2015 & 7 & 2006 & 6 \\
    Washington & - & - & 2012 & 1 \\
    West Virginia & 2012 & 6 & 2013 & 5 \\
    Wisconsin & 2017 & 4 & 2013 & 6 \\
    Wyoming & - & - & 2013 & 7 \\

\end{longtable}

\newpage

\begin{figure}[htbp]
    \centering
    \includegraphics[width=0.8\textwidth]{kaitz_percentiles.png}
    \caption{Only treated states are included in the sample. The figure shows the distribution of the Kaitz-\(\rho\) index across counties in the United States for the year a Must Query PDMP is implemented in their states. As lower percentiles from the wage distribution are considered, the minimum wage is more binding.}
    \label{fig:kaitz_perc}
\end{figure}

\newpage

\begin{figure}[htbp]
    \centering
    \includegraphics[width=0.8\textwidth]{pmq10_lab_for_rate.png}
    \caption{}
    \label{fig:lab_for_rate}
\end{figure}

\newpage

\begin{figure}[htbp]
    \centering
    \includegraphics[width=0.8\textwidth]{pmq10_lab_for_rate_ta.png}
    \caption{}
    \label{fig:lab_for_rate_ta}
\end{figure}

\newpage

\begin{figure}[htbp]
    \centering
    \includegraphics[width=0.8\textwidth]{pmq10_prescriptions.png}
    \caption{}
    \label{fig:prescriptions}
\end{figure}

\newpage

\begin{figure}[htbp]
    \centering
    \includegraphics[width=0.8\textwidth]{pmq10_unemp_rate.png}
    \caption{}
    \label{fig:unemp_rate}
\end{figure}

\newpage

\begin{figure}[htbp]
    \centering
    \includegraphics[width=0.8\textwidth]{pmq10_unemp_rate_ta.png}
    \caption{}
    \label{fig:unemp_rate_ta}
\end{figure}

\newpage

\begin{figure}[htbp]
    \centering
    \includegraphics[width=0.8\textwidth]{pmq10_heroin.png}
    \caption{}
    \label{fig:drugs}
\end{figure}

\newpage

\begin{table}[ht]
    \centering
    \begin{tabular}{lcccc}
    \toprule
     & \textbf{Estimate} & \textbf{Std. Error} & \textbf{t value} & \textbf{Pr($>|t|$)} \\
    \midrule
    Intercept  & 5.600 & 2.135 & 2.623 & 0.0211 * \\
    Kaitz-0.10 & 9.613 & 5.021 & 1.914 & 0.0778 . \\
    \midrule
    \multicolumn{5}{l}{\textit{Signif. codes:  0 '***' 0.001 '**' 0.01 '*' 0.05 '.' 0.1 ' ' 1}} \\
    \midrule
    \multicolumn{5}{l}{\textbf{Residual standard error}: 1.95 on 13 degrees of freedom} \\
    \multicolumn{5}{l}{\textbf{Multiple R-squared}: 0.2199, \textbf{Adjusted R-squared}: 0.1599} \\
    \multicolumn{5}{l}{\textbf{F-statistic}: 3.665 on 1 and 13 DF, \textbf{p-value}: 0.07783} \\
    \bottomrule
    \end{tabular}
    \caption{Regression Results: Heroin Overdose Deaths}
    \label{tab:regression_results_heroin}
\end{table}

\begin{table}[ht]
    \centering
    \begin{tabular}{lcccc}
    \toprule
     & \textbf{Estimate} & \textbf{Std. Error} & \textbf{t value} & \textbf{Pr($>|t|$)} \\
    \midrule
    Intercept  & -0.4800 & 0.8021 & -0.598 & 0.560 \\
    Kaitz-0.10 & -0.6460 & 1.8900 & -0.342 & 0.738 \\
    \midrule
    \multicolumn{5}{l}{\textit{Signif. codes:  0 '***' 0.001 '**' 0.01 '*' 0.05 '.' 0.1 ' ' 1}} \\
    \midrule
    \multicolumn{5}{l}{\textbf{Residual standard error}: 0.7576 on 13 degrees of freedom} \\
    \multicolumn{5}{l}{\textbf{Multiple R-squared}: 0.008906, \textbf{Adjusted R-squared}: -0.06733} \\
    \multicolumn{5}{l}{\textbf{F-statistic}: 0.1168 on 1 and 13 DF, \textbf{p-value}: 0.738} \\
    \bottomrule
    \end{tabular}
    \caption{Regression Results: Methadone Overdose Deaths}
    \label{tab:regression_results_methadone}
\end{table}

\newpage

\begin{table}[ht]
    \centering
    \begin{tabular}{lcccc}
    \toprule
     & \textbf{Estimate} & \textbf{Std. Error} & \textbf{t value} & \textbf{Pr($>|t|$)} \\
    \midrule
    Intercept  & -3.486 & 2.530 & -1.378 & 0.192 \\
    Kaitz-0.10 & -7.537 & 5.949 & -1.267 & 0.227 \\
    \midrule
    \multicolumn{5}{l}{\textit{Signif. codes:  0 '***' 0.001 '**' 0.01 '*' 0.05 '.' 0.1 ' ' 1}} \\
    \midrule
    \multicolumn{5}{l}{\textbf{Residual standard error}: 2.31 on 13 degrees of freedom} \\
    \multicolumn{5}{l}{\textbf{Multiple R-squared}: 0.1099, \textbf{Adjusted R-squared}: 0.04143} \\
    \multicolumn{5}{l}{\textbf{F-statistic}: 1.605 on 1 and 13 DF, \textbf{p-value}: 0.2274} \\
    \bottomrule
    \end{tabular}
    \caption{Regression Results: Other opioids Overdose Deaths}
    \label{tab:regression_results_otheropioids}
\end{table}

\begin{table}[ht]
    \centering
    \begin{tabular}{lcccc}
    \toprule
     & \textbf{Estimate} & \textbf{Std. Error} & \textbf{t value} & \textbf{Pr($>|t|$)} \\
    \midrule
    Intercept  & 4.944 & 5.995 & 0.825 & 0.424 \\
    Kaitz-0.10 & 8.056 & 14.097 & 0.571 & 0.577 \\
    \midrule
    \multicolumn{5}{l}{\textit{Signif. codes:  0 '***' 0.001 '**' 0.01 '*' 0.05 '.' 0.1 ' ' 1}} \\
    \midrule
    \multicolumn{5}{l}{\textbf{Residual standard error}: 5.475 on 13 degrees of freedom} \\
    \multicolumn{5}{l}{\textbf{Multiple R-squared}: 0.0245, \textbf{Adjusted R-squared}: -0.05053} \\
    \multicolumn{5}{l}{\textbf{F-statistic}: 0.3266 on 1 and 13 DF, \textbf{p-value}: 0.5774} \\
    \bottomrule
    \end{tabular}
    \caption{Regression Results: synthetic opioids Overdose Deaths}
    \label{tab:regression_results_synthetic}
\end{table}

\newpage

\section*{Appendix}

\begin{figure}[htbp]
    \centering
    \includegraphics[width=0.8\textwidth]{pmq10_lab_for_rate_p.png}
    \caption{}
    \label{fig:lab_for_rate_p}
\end{figure}

\begin{figure}[htbp]
    \centering
    \includegraphics[width=0.8\textwidth]{pmq10_lab_for_rate_comp6.png}
    \caption{}
    \label{fig:lab_for_rate_comp6}
\end{figure}

\begin{figure}[htbp]
    \centering
    \includegraphics[width=0.8\textwidth]{pmq10_lab_for_rate_comp24.png}
    \caption{}
    \label{fig:lab_for_rate_comp24}
\end{figure}

\begin{figure}[htbp]
    \centering
    \includegraphics[width=0.8\textwidth]{pmq10_unemp_rate_p.png}
    \caption{}
    \label{fig:unemp_rate_p}
\end{figure}

\begin{figure}[htbp]
    \centering
    \includegraphics[width=0.8\textwidth]{pmq10_unemp_rate_comp6.png}
    \caption{}
    \label{fig:unemp_rate_comp6}
\end{figure}

\begin{figure}[htbp]
    \centering
    \includegraphics[width=0.8\textwidth]{pmq10_unemp_rate_comp24.png}
    \caption{}
    \label{fig:unemp_rate_comp24}
\end{figure}

\end{document}
