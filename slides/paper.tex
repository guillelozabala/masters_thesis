\documentclass[12pt,a4paper]{article}
\usepackage[margin=1in]{geometry}
\usepackage{graphicx}
\usepackage{amsmath,amssymb,amsfonts,bbm}
\usepackage{comment}
\usepackage{hyperref}
\usepackage[backend=bibtex,style=authoryear]{biblatex}

\addbibresource{refs.bib}

\hypersetup{
    colorlinks=true,
    linkcolor=blue,
    filecolor=magenta,      
    urlcolor=cyan,
    pdftitle={paper},
    pdfpagemode=FullScreen,
}

\setlength{\abovedisplayskip}{3pt}
\setlength{\belowdisplayskip}{3pt}

\setlength\parindent{0pt}
\addtolength{\parskip}{2mm}

\title{The Opioid Crisis: State Regulations and Labor Market Outcomes \\
[2ex] \large CEMFI}
\author{Guillermo Martínez Martínez \\ [2ex] Advisor: Tom Zohar}
\date{\today}

\begin{document}

\maketitle

\begin{abstract}
    \noindent
    While the effects of opioid misuse on the labor market have been studied in detail, little is known about how labor market regulations, particularly the minimum wage, shape this relationship. Do minimum wage rates affect the relationship between opioid misuse and labor market outcomes? To tackle this question, I use the staggered rollout of Must Query Prescription Drug Monitoring Programs (PDMPs) as a source of variation. These policies restrict prescription opioids for those considered to be misusing them, forcing individuals to either quit consumption or switch to illegal alternatives. The benefits of quitting opioids depend on the labor market opportunities available to those who withdraw. Through their negative impact on labor demand, high (binding) minimum wages can reduce the incentives to avoid opioid consumption. I find that labor force participation and unemployment rates increased only where the minimum wage was not binding, while a surge in heroin consumption followed where it was. These results highlight the importance of considering labor market regulations when studying the effects of the opioid crisis on the labor market.
\end{abstract}

% Pablo Muñoz: clarificar a que me refiero con binding 

\newpage

\section*{Introduction}

For the past two decades, the United States has been facing the worst public health crisis in its recent history.
The so-called opioid epidemic, characterized by a rising number of deaths due to opioid overdose, has caused more than 680,000 fatalities between 1999 and 2022, mostly among non-college-educated Americans.
This situation has prompted economists to study the crisis's effects on the labor market.
However, little is known about how labor market regulations have shaped this relationship.
Minimum wage rates emerge as a natural candidate for studying this relationship.
These policies primarily affect the same subpopulation of low-educated workers and have increased with varying intensity during the same period.
Do minimum wage rates affect the relationship between opioid misuse and labor market outcomes?

To measure this relationship, I use the staggered rollout of Must Query Prescription Drug Monitoring Programs (PDMPs) across states.
These programs require authorized users to check a patient's prescription history before prescribing opioids.
As these measures effectively combat opioid misuse, they force those who are addicted at the time the law is passed to either quit opioid consumption or switch to illegal alternatives.
The incentives to choose one option over the other depend on the labor market opportunities available to those who withdraw: quitting opioids, although costly and often painful, offers benefits not only in better health outcomes but also in job prospects. 
Hence, for those facing a minimum wage high enough to lower labor demand (binding), the benefits of quitting opioids might be lower than for those with greater job opportunities.

If this argument holds, we should observe an increase in job-searching behavior where the minimum wage is not binding and an increase in illegal opioid consumption where it is.
To test these hypotheses, I regress county-level labor force participation and unemployment rates on the rollout of PDMPs using an individual effects event study following \textcite{arkhangelsky2024flexible}.
Critical for identification, I assume that the passing of these state-level policies is exogenous from the point of view of the counties, after controlling for the number of overdose deaths, among other variables.
I then compare the distribution of coefficients to how binding the minimum wage rates are in the different counties.
In a second stage, I compare the impact of prescription policies on overdose deaths for different opioids using the same measure of minimum wage bindingness.

The results are striking in several dimensions.
First, I find that policies aimed at stopping opioid misuse increased participation rates only where the minimum wage was not binding.
Secondly, the increase was short-lived (less than two years), but the impact on unemployment rates was persistent.
Moreover, their impact on prescription rates seem to be the main, but not the only, driver of these results.
Finally, prescription policies had no labor market consequences where where the minimum wage was binding, but their impact on heroin deaths was increasing in minimum wage bindingness.

These results shed light on how labor market regulations in general, and minimum wage rates in particular, give rise to heterogeneities in the effects of the opioid crisis on the labor market.
Moreover, they highlight the importance of blah blah.

\section*{Literature review}

This paper is located at the intersection of three strands of literature.

Large and significant negative impact of prescriptions and misuse on labor outcomes: Aliprantis, Fee, and Schweitzer (2023), Beheshti (2022), Greenwood, Guner and Kopecky (2022) % , Harris et al. (2018)
Prescription drugs regulations can be effective in reducing opioid misuse: Buchmueller and Carey (2018), Ukert and Polsky (2023)
Positive relation between legal opioid misuse and heroin deaths: Alpert, Powell, and Pacula (2018), Cicero, Ellis, and Surratt (2012), Cicero and Ellis (2015)

\textbf{Contribution}: shedding light on how the labor effects of the Opioid Crisis have been affected by minimum wage rates

Interest in how labor market regulations interact with Substance Abuse Disorders

\begin{comment}

#### Negative Impact of Opioid Misuse on Labor Outcomes

Extensive research has documented the large and significant negative impact of opioid prescriptions and misuse on labor market outcomes. Aliprantis, Fee, and Schweitzer (2023) highlight how opioid addiction leads to substantial declines in employment rates and labor force participation. Similarly, Beheshti (2022) finds that opioid misuse is linked to reduced work hours and earnings, particularly among low-educated workers. Greenwood, Guner, and Kopecky (2022) show that regions with high opioid prescription rates experience notable reductions in economic productivity and workforce stability. Harris et al. (2018) further corroborate these findings, demonstrating the broad economic costs associated with opioid addiction, including increased unemployment and decreased job-seeking behavior.

#### Effectiveness of Prescription Drug Regulations

Research also indicates that prescription drug regulations can effectively reduce opioid misuse. Buchmueller and Carey (2018) provide evidence that policies limiting opioid prescriptions lead to substantial declines in opioid abuse and related health issues. Ukert and Polsky (2023) build on this work by showing that state-level interventions, such as Prescription Drug Monitoring Programs (PDMPs), significantly decrease the availability of prescription opioids and reduce rates of opioid misuse. These studies collectively suggest that regulatory measures can be an essential tool in combating the opioid crisis.

#### Relationship Between Legal Opioid Misuse and Heroin Deaths

The transition from legal opioid misuse to heroin use has been well-documented, with several studies illustrating a positive relationship between the two. Alpert, Powell, and Pacula (2018) find that restrictions on prescription opioids often lead individuals to seek out heroin as a cheaper and more accessible alternative, resulting in increased heroin overdose deaths. Cicero, Ellis, and Surratt (2012) also identify this substitution effect, noting a significant rise in heroin use following efforts to curtail prescription opioid abuse. Further research by Cicero and Ellis (2015) supports these findings, emphasizing the need for comprehensive approaches that address both prescription and illicit opioid markets to effectively reduce overall opioid-related mortality.

Feel free to adjust the details or add more context based on the specific focus and needs of your paper!
\end{comment}


\section*{Model}

An addicted worker \(i\) faces at time \(t\):

\begin{itemize}
    \item A health/performance cost of opioid consumption: \(O_{it}\)
    \item This cost depends on legal opioids consumption \(l_{it}\) and black market opioids consumption \(h_{it}\) (say, heroin)
    \item Which opioid is consumed depends on the availability of legal opioids \(z_{it}\) (\(= 1\) if prescription law is in place)
    \item I assume that opioid misuse impedes the worker's ability to work
\end{itemize}

The decision rule of opioid consumption is hence given by:

\[
    O_{it} = z_{t}\left(1-q_{it}\right)h_{it} + (1-z_{t})l_{it}
\]

where \(q_{it} = 1\) if she quits opioids. The worker quits if
        
\[
    q_{it} = \mathbbm{1}\{\text{Exp. utility of quitting} \geq \text{Disutility of quitting}\}
\]

Quitting and looking for a job are equivalent.The market has a job finding rate

\[
    p_{it} = m_{it}/u_{it} 
\]

where \(m_{it}\) is the number of matches and \(u_{it}\) is the number of unemployed workers. Matches are a non-decreasing function of vacancies \(v_{it}\)

\[
    m_{it} = M(v_{it}, u_{it})
\]

where \(M\left(\cdot \right) \) is a matching function. I assume that more vacancies are open where the minimum wage is less binding (\cite{flinn2010}). It follows that

\[
    p_{w_{it} > w^{min}_t} \geq p_{w_{it} = w^{min}_t}
\]

If \(w^{min}_t\) is not binding

\[
    q_{w_{it} > w^{min}_t} = \mathbbm{1}\{p_{w_{it} > w^{min}_t}\cdot u(w_{it}L_{it}) \geq \mathcal{C}^{quit}\}
\]

Otherwise

\[
    q_{w_{it} = w^{min}_t} = \mathbbm{1}\{p_{w_{it} = w^{min}_t}\cdot u(w^{min}_t L_{it}) \geq \mathcal{C}^{quit}\} 
\]

As long as

\[
    p_{w_{it} > w^{min}_t}\cdot u(w_{it}L_{it}) > \mathcal{C}^{quit} > p_{w_{it} = w^{min}_t}\cdot u(w^{min}_t L_{it})
\]

we should see (after \(z_{t} = 1\)) an increase in participation rates where the minimum wage is less binding (\(O_{it} = 0\)); an increase in illegal opioid consumption where the minimum wage is more binding (\(O_{it} = h_{it}\)).

\section*{Data}

US Bureau of Labor Statistics (BLS):
LAUS Database: County-level labor market statistics (1990-2023) % Local Area Unemployment Statistics
OEWS Database: National industry-specific occupations and wages statistics (1997-2023) % Occupational Employment and Wage Statistics

Eckert et al. (2020): County-level industry-specific employment (1975-2016)

USDOL's Changes in Basic Minimum Wages (1968-2023)

US Census Bureau's Population Estimates Program (PEP) : County-level population and demographics (1969-2020)

DEA's ARCOS : Opioid distribution data (2006-2014)

Horwitz et al. (2020) : Prescription Drug Monitoring Programs (1990-2019)

Centers for Disease Control (CDC):
National Vital Statistics System: Total drug poisoning mortality by county (2003-2021)
Multiple Cause of Death: Overdose deaths at the state-year level, for different drugs (1999-2020)

Current working sample:
LMOs, PDMPs, Minimum Wage rates and demographics for counties of 50 States + DC, 1998-2019
County-level industry-weighted wage distribution and death mortality for counties of 50 States + DC, 2003-2016

Prescription Drug Monitoring Programs (PDMPs): electronic database that tracks controlled substance prescriptions
Modern System PDMPs: the electronic database becomes accessible to any authorized user (eg, physician, pharmacist, or member of law enforcement)
Must Query PDMPs: the law mandates the prescriber to check the database before prescribing a listed opioid


\section*{Results}

\section*{Conclusions}

\printbibliography

\end{document}